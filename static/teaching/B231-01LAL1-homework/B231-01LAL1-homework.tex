\documentclass{article}

\usepackage{polyglossia}

\usepackage{amsmath}
\usepackage{amssymb}
\usepackage{exercise}
\usepackage[a4paper, margin = 2cm]{geometry}
\usepackage{microtype}
\usepackage{systeme}

\setdefaultlanguage{czech}

\title{Domácí úkoly LAL1}

\sysdelim..

\newcommand{\mathbasis}{\ensuremath{\mathcal}}
\newcommand{\mathfield}{\ensuremath{\mathbb}}
\newcommand{\mathmat}{\ensuremath{\mathbb}}
\newcommand{\mathspace}{\ensuremath{\mathcal}}
\newcommand{\mathvec}{\ensuremath{\vec}}
\newcommand{\subspace}{\ensuremath{\subset \subset}}

\begin{document}

\maketitle

\begin{Exercise}[name=Úkol]
	Uvažujme prostor \( \mathfield{C}^{2, 2} \), jeho standardní bázi \( \mathbasis{E} = \left( \mathmat{E}_1, \mathmat{E}_2, \mathmat{E}_3, \mathmat{E}_4 \right) \) a báze \( \mathbasis{X} = \left( \mathmat{X}_1, \mathmat{X}_2, \mathmat{X}_3, \mathmat{X}_4 \right) \) a \( \mathbasis{Y} = \left( \mathmat{Y}_1, \mathmat{Y}_2, \mathmat{Y}_3, \mathmat{Y}_4 \right) \), kde
	\[
		\mathmat{X}_1 = \begin{pmatrix} 1 & 2 \\ 0 & 1 \end{pmatrix} \qquad
		\mathmat{X}_2 = \begin{pmatrix} 0 & 0 \\ 3 & -1 \end{pmatrix} \qquad
		\mathmat{X}_3 = \begin{pmatrix} 2 & 3 \\ 3 & 2 \end{pmatrix} \qquad
		\mathmat{X}_4 = \begin{pmatrix} 1 & 0 \\ 0 & 2 \end{pmatrix}
	\]
	a
	\[
		\left( \mathmat{Y}_1 \right)_\mathbasis{X} = \begin{pmatrix} 1 \\ 1 \\ 0 \\ 1 \end{pmatrix} \qquad
		\left( \mathmat{Y}_2 \right)_\mathbasis{X} = \begin{pmatrix} 0 \\ 1 \\ 0 \\ 2 \end{pmatrix} \qquad
		\left( \mathmat{Y}_3 \right)_\mathbasis{X} = \begin{pmatrix} 1 \\ 0 \\ 0 \\ 0 \end{pmatrix} \qquad
		\left( \mathmat{Y}_4 \right)_\mathbasis{X} = \begin{pmatrix} -4 \\ -3 \\ -2 \\ -1 \end{pmatrix} \text{.}
	\]
	Dále definujme matice \( \mathmat{A} \in \mathfield{C}^{2, 2} \) a \( \mathmat{B} \in \mathfield{C}^{2, 2} \) jako
	\[
		\left( \mathmat{A} \right)_\mathbasis{Y} = \begin{pmatrix} 1 \\ 2 \\ 3 \\ 4 \end{pmatrix} \qquad
		\left( \mathmat{B} \right)_\mathbasis{E} = \begin{pmatrix} 10 \\ 18 \\ 15 \\ 7 \end{pmatrix}\text{.}
	\]
	\Question Najděte \( \mathmat{A} \).
	\Question Najděte \( \left( \mathmat{B} \right)_\mathbasis{Y} \).
	\Question Najděte \( \left( \mathmat{E}_1 \right)_\mathbasis{X} \).
	\Question Najděte \( \left( \mathmat{E}_2 \right)_\mathbasis{Y} \).
	\Question Najděte bázi prostoru \( \mathfield{C}^{2, 2} \) obsahující \( \mathmat{X}_1, \mathmat{Y}_2, \mathmat{E}_3 \).
\end{Exercise}

\begin{Exercise}[name=Úkol]
	Nechť \( P \subspace \mathfield{R}^5 \) a \( Q \subspace \mathfield{R}^5 \), kde
	\[
		P = \left[
			\begin{pmatrix} 1 \\ 2 \\ 0 \\ 1 \\ 2 \end{pmatrix},
			\begin{pmatrix} 1 \\ 0 \\ 0 \\ 0 \\ 1 \end{pmatrix},
			\begin{pmatrix} 0 \\ 1 \\ 1 \\ 1 \\ 0 \end{pmatrix}
		\right]_\lambda
	\]
	a
	\[
		Q = \left\{
			\begin{pmatrix} x_1 \\ x_2 \\ x_3 \\ x_4 \\ x_5 \end{pmatrix} \in \mathfield{R}^5
			\: \middle| \:
			\systeme{
				x_1 - x_2 +x_3 + 3x_4 -x_5 = 0,
				-x_1 + 2x_3 - x_4 + x_5 = 0,
				3x_1 - 2x_2 + 7x_4 - 3x_5 = 0
			}
		\right\} \text{.}
	\]
	\Question Platí \( P \subspace Q \) ?
	\Question Najděte dimenzi a bázi \( P \cap Q \), \( P \cup Q \) a \( P + Q \), pokud jsou to dobře definované prostory.
	\Question Najděte doplněk \( Q \) do \( \mathfield{R}^5 \).
\end{Exercise}

\newpage

\begin{Exercise}[name=Úkol]
	Nechť \( A : \mathfield{C}^2_\mathfield{R} \to \mathspace{P}_5 \) (kde \( \mathspace{P}_5 \) označíme prostor reálných polynomů stupně nejvýše 4 s přidáním nulového polynomu) takové, že \( \forall \alpha, \beta, \gamma, \delta, t \in \mathfield{R} \) platí
	\[
		\begin{aligned}
			 \left( A \begin{pmatrix} \alpha + \beta i \\ \gamma + \delta i \end{pmatrix} \right) \left( t \right) = &\left( \alpha + \beta \right) + \left( \alpha + 4\beta - 2\gamma - \delta \right) t + \\
			+ &\left( \gamma - \beta \right) t^2 + \left( 2\alpha + \gamma + \delta \right) t^3 + \left( \alpha + \delta \right) t^4 \text{.}
		\end{aligned}
	\]
	\Question Ukažte, že \( A \) je lineární zobrazení.
	\Question Najděte hodnost, defekt a jádro \( A \).
	\Question Určete, zda je \( A \) monomorfní, epimorfní a izomorfní.
	\Question Popište množinu všech řešení \( \mathvec{x} \) rovnice
	\[
		\forall t \in \mathfield{R} \quad \left( A \mathvec{x} \right) \left( t \right) = 3 - t + t^2 + 9t^3 + 5t^4 \text{.}
	\]
\end{Exercise}

\end{document}
